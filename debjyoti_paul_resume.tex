%%%%%%%%%%%%%%%%%%%%%%%%%%%%%%%%%%%%%%%%%%%%%%%%%%%%%%%%%%%%%%%%%%%%%%%%
%%%%%%%%%%%%%%%%%%%%%% Simple LaTeX CV Template %%%%%%%%%%%%%%%%%%%%%%%%
%%%%%%%%%%%%%%%%%%%%%%%%%%%%%%%%%%%%%%%%%%%%%%%%%%%%%%%%%%%%%%%%%%%%%%%%

%%%%%%%%%%%%%%%%%%%%%%%%%%%%%%%%%%%%%%%%%%%%%%%%%%%%%%%%%%%%%%%%%%%%%%%%
%% NOTE: If you find that it says                                     %%
%%                                                                    %%
%%                           1 of ??                                  %%
%%                                                                    %%
%% at the bottom of your first page, this means that the AUX file     %%
%% was not available when you ran LaTeX on this source. Simply RERUN  %%
%% LaTeX to get the ``??'' replaced with the number of the last page  %%
%% of the document. The AUX file will be generated on the first run   %%
%% of LaTeX and used on the second run to fill in all of the          %%
%% references.                                                        %%
%%%%%%%%%%%%%%%%%%%%%%%%%%%%%%%%%%%%%%%%%%%%%%%%%%%%%%%%%%%%%%%%%%%%%%%%

%%%%%%%%%%%%%%%%%%%%%%%%%%%% Document Setup %%%%%%%%%%%%%%%%%%%%%%%%%%%%

% Don't like 10pt? Try 11pt or 12pt
\documentclass[8pt]{article}

% This is a helpful package that puts math inside length specifications
\usepackage{calc}
\usepackage{comment}
\usepackage[none]{hyphenat}
% Simpler bibsection for CV sections
% (thanks to natbib for inspiration)
\makeatletter
\newlength{\bibhang}
\setlength{\bibhang}{1em}
\newlength{\bibsep}
 {\@listi \global\bibsep\itemsep \global\advance\bibsep by\parsep}
\newenvironment{bibsection}%
        {\vspace{-\baselineskip}\begin{list}{}{%
       \setlength{\leftmargin}{\bibhang}%
       \setlength{\itemindent}{-\leftmargin}%
       \setlength{\itemsep}{\bibsep}%
       \setlength{\parsep}{\z@}%
        \setlength{\partopsep}{0pt}%
        \setlength{\topsep}{0pt}}}
        {\end{list}\vspace{-.6\baselineskip}}
\makeatother

% Layout: Puts the section titles on left side of page
\reversemarginpar

%
%         PAPER SIZE, PAGE NUMBER, AND DOCUMENT LAYOUT NOTES:
%
% The next \usepackage line changes the layout for CV style section
% headings as marginal notes. It also sets up the paper size as either
% letter or A4. By default, letter was used. If A4 paper is desired,
% comment out the letterpaper lines and uncomment the a4paper lines.
%
% As you can see, the margin widths and section title widths can be
% easily adjusted.
%
% ALSO: Notice that the includefoot option can be commented OUT in order
% to put the PAGE NUMBER *IN* the bottom margin. This will make the
% effective text area larger.
%
% IF YOU WISH TO REMOVE THE ``of LASTPAGE'' next to each page number,
% see the note about the +LP and -LP lines below. Comment out the +LP
% and uncomment the -LP.
%
% IF YOU WISH TO REMOVE PAGE NUMBERS, be sure that the includefoot line
% is uncommented and ALSO uncomment the \pagestyle{empty} a few lines
% below.
%

%% Use these lines for letter-sized paper
\usepackage[paper=letterpaper,
            %includefoot, % Uncomment to put page number above margin
            marginparwidth=.85in,     % Length of section titles
            marginparsep=.05in,       % Space between titles and text
            margin=0.55in,               % 1 inch margins
            includemp]{geometry}

%% Use these lines for A4-sized paper
%\usepackage[paper=a4paper,
%            %includefoot, % Uncomment to put page number above margin
%            marginparwidth=30.5mm,    % Length of section titles
%            marginparsep=1.5mm,       % Space between titles and text
%            margin=25mm,              % 25mm margins
%            includemp]{geometry}

%% More layout: Get rid of indenting throughout entire document
\setlength{\parindent}{0in}

%% This gives us fun enumeration environments. compactitem will be nice.
\usepackage{paralist}

%% Reference the last page in the page number
%
% NOTE: comment the +LP line and uncomment the -LP line to have page
%       numbers without the ``of ##'' last page reference)
%
% NOTE: uncomment the \pagestyle{empty} line to get rid of all page
%       numbers (make sure includefoot is commented out above)
%
\usepackage{fancyhdr,lastpage}
\pagestyle{fancy}
%\pagestyle{empty}      % Uncomment this to get rid of page numbers
\fancyhf{}\renewcommand{\headrulewidth}{0pt}
\fancyfootoffset{\marginparsep+\marginparwidth}
\newlength{\footpageshift}
\setlength{\footpageshift}
          {0.5\textwidth+0.5\marginparsep+0.5\marginparwidth-2in}
\lfoot{\hspace{\footpageshift}%
       \parbox{4in}{\, \hfill %
                    \arabic{page} of \protect\pageref*{LastPage} % +LP
%                    \arabic{page}                               % -LP
                    \hfill \,}}

% Finally, give us PDF bookmarks
\usepackage{color,hyperref}
\definecolor{darkblue}{rgb}{0.0,0.0,0.3}
\hypersetup{colorlinks,breaklinks,
            linkcolor=darkblue,urlcolor=darkblue,
            anchorcolor=darkblue,citecolor=darkblue}

%%%%%%%%%%%%%%%%%%%%%%%% End Document Setup %%%%%%%%%%%%%%%%%%%%%%%%%%%%


%%%%%%%%%%%%%%%%%%%%%%%%%%% Helper Commands %%%%%%%%%%%%%%%%%%%%%%%%%%%%

% The title (name) with a horizontal rule under it
%
% Usage: \makeheading{name}
%
% Place at top of document. It should be the first thing.
\newcommand{\makeheading}[1]%
        {\hspace*{-\marginparsep minus \marginparwidth}%
         \begin{minipage}[t]{\textwidth+\marginparwidth+\marginparsep}%
                {\large \bfseries #1}\\[-0.15\baselineskip]%
                 \rule{\columnwidth}{1pt}%
         \end{minipage}}

% The section headings
%
% Usage: \section{section name}
%
% Follow this section IMMEDIATELY with the first line of the section
% text. Do not put whitespace in between. That is, do this:
%
%       \section{My Information}
%       Here is my information.
%
% and NOT this:
%
%       \section{My Information}
%
%       Here is my information.
%
% Otherwise the top of the section header will not line up with the top
% of the section. Of course, using a single comment character (%) on
% empty lines allows for the function of the first example with the
% readability of the second example.
\renewcommand{\section}[2]%
        {\pagebreak[2]\vspace{0.7\baselineskip}%
         \phantomsection\addcontentsline{toc}{section}{#1}%
         \hspace{0in}%
         \marginpar{
         \raggedright \scshape #1}#2}

% An itemize-style list with lots of space between items
\newenvironment{outerlist}[1][\enskip\textbullet]%
        {\begin{itemize}[#1]}{\end{itemize}%
         \vspace{-.6\baselineskip}}

% An environment IDENTICAL to outerlist that has better pre-list spacing
% when used as the first thing in a \section
\newenvironment{lonelist}[1][\enskip\textbullet]%
        {\vspace{-\baselineskip}\begin{list}{#1}{%
        \setlength{\partopsep}{0pt}%
        \setlength{\topsep}{0pt}}}
        {\end{list}\vspace{-.6\baselineskip}}

% An itemize-style list with little space between items
\newenvironment{innerlist}[1][\enskip\textbullet]%
        {\begin{compactitem}[#1]}{\end{compactitem}}

% An environment IDENTICAL to innerlist that has better pre-list spacing
% when used as the first thing in a \section
\newenvironment{loneinnerlist}[1][\enskip\textbullet]%
        {\vspace{-\baselineskip}\begin{compactitem}[#1]}
        {\end{compactitem}\vspace{-.6\baselineskip}}

% To add some paragraph space between lines.
% This also tells LaTeX to preferably break a page on one of these gaps
% if there is a needed pagebreak nearby.
\newcommand{\blankline}{\quad\pagebreak[2]}

% Uses hyperref to link DOI
\newcommand\doilink[1]{\href{http://dx.doi.org/#1}{#1}}
\newcommand\doi[1]{doi:\doilink{#1}}


%%%%%%%%%%%%%%%%%%%%%%%% End Helper Commands %%%%%%%%%%%%%%%%%%%%%%%%%%%

%%%%%%%%%%%%%%%%%%%%%%%%% Begin CV Document %%%%%%%%%%%%%%%%%%%%%%%%%%%%

\begin{document}
\makeheading{\Large{Debjyoti Paul}}

\section{Contact Information}
%
% NOTE: Mind where the & separators and \\ breaks are in the following
%       table.
%
% ALSO: \rcollength is the width of the right column of the table
%       (adjust it to your liking; default is 1.85in).
%
\newlength{\rcollength}\setlength{\rcollength}{2.5in}%
%

\begin{tabular}[t]{@{}p{\textwidth-\rcollength}p{\rcollength}}
130 S 800 E APT 101  &  +1 385 313-7219  \\
 Salt Lake City   &  \href{mailto:deb@cs.utah.edu}{deb@cs.utah.edu}\\
 Pin: 84102  &  \href{mailto:debjyoti.paul@utah.edu}{debjyoti.paul@utah.edu} \\ 
        University of Utah & \href{https://github.com/debjyoti385}{\url{http://github.com/debjyoti385}} \\ 
        USA     & \href{http://www.cs.utah.edu/~deb}{\url{http://www.cs.utah.edu/\~deb}}\\
\end{tabular}


\section{Research Interests}
%
Large Scale Social Media Data Analytics, Machine Learning and Data Mining techniques, Deep Learning, Bayesian Learning, Text Analytics,  Indexing techniques, Data Visualization. 
%

\section{Education}
%
\href{http://www.cs.utah.edu/}{\textbf{University of Utah, School of Computing}},
Salt Lake City, UT, USA
\begin{outerlist}

\item[] PhD. Student,
        \href{http://cs.utah.edu/}
             {Computer Science \& Engineering},
        Fall 2015- \textit{Current}
        \begin{innerlist}
%        \item[] Thesis Title: \emph{Multi-constraint Job scheduling problem in Grid}
        \item[] Current Research Advisor:
              \href{http://www.cs.utah.edu/~feifei/}
                   {Professor Feifei Li}
%        \item[] Research: Efficient scheduling strategy of Jobs on Grid to achieve maximum utilization of resources with energy efficient solution and maintaining hard and soft constraints along with SLAs.
    \item[] GPA: 3.93/4.0
\\       
        \end{innerlist}
\end{outerlist} 


\href{http://www.iitk.ac.in/}{\textbf{Indian Institute of Technology Kanpur}},
Kanpur, UP, INDIA
\begin{outerlist}

\item[] M.Tech.,
        \href{http://cse.iitk.ac.in/}
             {Computer Science \& Engineering},
             2011-2013
        \begin{innerlist}
        \item[] Thesis Title: \emph{Multi-constraint Job scheduling problem in Grid}
        \item[] Research Adviser:
                \href{}
                   {Professor Late Sanjeev K Aggarwal}
        \item[] Research: Efficient scheduling strategy of Jobs on Grid with energy efficient solution for defined SLAs.
    \item[] CGPA: 8.67/10 (\emph{Rank:  3})
	\\       
        \end{innerlist}
\end{outerlist} 

\href{http://www.wbut.ac.in/}{\textbf{West Bengal University of Technology}},
Salt Lake, Kolkata, WB, INDIA
\begin{outerlist}
\item[] B.Tech.,
        Computer Science \& Engineering, 2007-2011
        \begin{innerlist}
        \item[] College: \href{http://www.iemcal.com/}{\emph{Institute of Enginering and Management}}
        \item[] CGPA: 8.93/10 (\emph{Rank:} $< 10$)
             \end{innerlist}

\end{outerlist}

\section{Work Experience}
\href{http://www.flipkart.com/}{\textbf{Flipkart}}, July 2013 - May 2015, Bangalore, INDIA
\begin{outerlist}
\item[] Software Developer in Data Platform ( Exceeds Expectations for performance in Jan-July 2014)
\begin{innerlist}
\item Data Platform is data bank of \href{http://www.flipkart.com}{\emph{flipkart.com}}, India's biggest e-commerce company
\item Experience in building Data Processing Platform (Ingestion-Processing-Visualization) and data warehousing. 
\item Practied Big data analytics in scalable environment, processing pipeline for batch and stream. Database administrator of HP Vertica.
\end{innerlist}
\end{outerlist}

\section{Theses}
M.Tech Thesis, Advisor: \href{http://www.cse.iitk.ac.in/users/ska}{Late Dr. Sanjeev Kumar Aggarwal}
\begin{outerlist}
\item[] \textbf{Multi-constraint Job scheduling problem in Grid} 
\begin{innerlist}
    \item The objective is to efficiently schedule Jobs on Grid to achieve maximum utilization of resources with energy efficient approach. Modelling real world computing \& storage grid on Multi-objective Evolutionary Algorithm satisfying hard constraints on jobs constraints, resources constraints, and soft constraints on cost, and energy consumption on NSGA-II.
\item Finding near Optimal Scheduling Solution: Local optimization using Pareto optimal front, global optimization by applying mutation on population on search space. Introduced job grouping technique with constraints to accumulate fine grained jobs which keeps processing time as low as possible.
\end{innerlist}
\end{outerlist}



%\textbf{[2010] Jadavpur University,} \emph{System Administration}, Trainee,\\
%    Kolkata, India
%    \begin{innerlist}
%    \item Practical experience with servers and networking devices and on various aspect of system administration
%    \end{innerlist}

\section{Publications} 
\\
\begin{bibsection}
\item \textbf{Debjyoti Paul}, Sanjeev K. Aggarwal, Multi-objective Evolution based Dynamic Job Scheduler in Grid, \emph{The 8th International Conference on Complex, Intelligent, and Software Intensive Systems (CISIS 2014)}, IEEE, July 2nd - 4th, 2014, Birmingham, UK.
    
\item  Manash Pal, Arnab Bhattacharya, \textbf{Debjyoti Paul}, RCached-tree: An Index Structure for Efficiently Answering Popular Queries, \emph{ACM International Conference on Information and Knowledge Management (CIKM 2013)}, Oct. 27 - Nov. 1, 2013, San Francisco, CA, USA. 

\item \textbf{Debjyoti Paul}, Sumana Basu, Punit Beriwal, IEEE Paper titled “Multilevel Security Protocol using Radio Frequency Identification (RFID)”, \emph{International Conference on Emerging Trends in Mathematics and Computer Applications–2010}, \emph{pg-}544-547, 2010, Sivakasi, TN, INDIA.   

\item \textbf{Debjyoti Paul}, Sumana Basu, Punit Beriwal, IEEE, \emph{Lightweight Security Enhancement Protocol for Radio Frequency
    Identification (RFID)” in SPSITM International Conference}, 2011, Kolkata, WB, INDIA       

\end{bibsection}

%\section{Summer Schools}\\
%\textbf{[2012] Microsoft Research}, \emph{Summer School on Distributed Computing}\\
%Bangalore, India
%        \begin{innerlist}
%        \item Familiarization with current research work on distributed computing\\
%            \end{innerlist}

%\section{Scholarship} 
%\begin{innerlist}
%\item GBP 1,700 from IIT Kanpur for Paper presentation in CISIS 2014, Birmingham, UK
%\item 96,000 INR per year from Ministry of Human Resource Development India during M.Tech.
%\item 2 years waiver of tuition fees at school for academic excellence in school.
%\end{innerlist}


%\section{Certifications} 
%\begin{innerlist}
%\item[\textbf{[2011]}] Linux system administration from School of Mobile Computing and Communication Jadavpur University.
%\item[\textbf{[2010]}] Microsoft certification on Database Fundamentals. 
%\end{innerlist}


\section{Projects \href{https://www.cs.utah.edu/~deb/research.php}{(for more..)}}
\begin{tabular}[t]{@{}p{\textwidth-\rcollength-180pt}p{\rcollength+180pt}}
        \href{http://estorm.org}{Compass: \hskip 2cm Generic Spatio-temporal Sentiment Analysis Framework \hskip 2cm (US Election 2016)} \hskip 1cm -\href{http://estorm.org}{\textit{http://estorm.org}} & \hspace{0.5cm} This project explores how Twitter can be used to analyze the popularity of Political Parties for the Presidential Election 2016. It is purely based on the data collected from Twitter. We provide unbiased analysis of popularity based on the sentiment analysis for each party (Republican and Democratic) at basic geospatial unit area which is `county' (not Electoral Vote region) is our case. We also present time-range analysis of the data. Also a Bursty Event Detection System is used to detect when surge of tweets happened. Our Bursty Event detection correctly identified the tweet bursts and associated with the event happened at that time. \\[0.1cm] 
         Event Aggregator & \hspace{0.5cm}  This project finds events from news articles, categorize them and gathers all articles talking about same event to a set. Each set of articles is referredas Event Entity. Information extraction process is applied to extract more information related to it. The code can process large scale data. The code will be made public after the paper is published.  \\[0.1cm] 
        \href{https://bitbucket.org/debjyotipaul385/quakeanalysis/wiki/Home}{QuakeAnalysis} -\href{https://goo.gl/tp3sEk}{\textit{https://goo.gl/tp3sEk}} & \hspace{0.5cm}  This project is based on the seismic activity across world which widely varies in characteristic and patterns. We have found some distinguish patterns among the seismic activities and present them in an insightful manner. Check \href{http://db03.cs.utah.edu:9000/}{demo} and wiki of \href{https://bitbucket.org/debjyotipaul385/quakeanalysis/wiki/Exploration}{Exploration} \& \href{https://bitbucket.org/debjyotipaul385/quakeanalysis/wiki/Analysis}{Analysis}\\[0.2cm]
        \href{https://musicatlas.github.io/}{MusicAtlas} -\href{https://goo.gl/UH54UP}{\textit{https://goo.gl/UH54UP}} & \hspace{0.5cm}  This website is designed for music lovers to learn about music based on countries. This is a unique tool to explore and analyze the trend of music based on time frame, genre and artists. Almost all data from 19th century to till date. Please make sure you allow SSL connection from api server via this \href{https://db03.cs.utah.edu:8181/api/country_track}{link} before accessing \href{https://musicatlas.github.io/musicatlas/}{@musicatlas}.\\[0.1cm]
        \href{https://github.com/debjyoti385/QuestionAnswerNLP}{QuestionAnswering} -\href{https://goo.gl/QGfW2y}{\textit{https://goo.gl/QGfW2y}} & \hspace{0.5cm}  Closed domain Question Answer System. The system has \emph{Recall} of  63\% and \emph{F-score} of 43\%.   \\[0.1cm]
        \href{http://debjyoti385.github.io/AirQuality/}{AirQuality} -\href{https://goo.gl/F8twhn}{\textit{https://goo.gl/F8twhn}}  & \hspace{0.5cm}  Goldman Sachs Hackathon - This analyze the air quality of Utah state from real world dataset. Data warehousing and cubing to enable interactive visualization framework, which helped in determining pattern of Pollution and its reason. \\
        RCachedTree -\href{https://doi.org/10.1145/2505515.2507817}{\textit{10.1145/2505515.2507817}} & \hspace{0.5cm}  The objective is to speed up performance of point, range and kNN search queries for frequent or popular queries in databases. We came up with a variant of the R-Tree indexing structure with some features like caching at each node for popular queries.\\ 
        \href{http://github.com/debjyoti385/intelliad}{Intelliad} -\href{https://goo.gl/UAnEdz}{\textit{https://goo.gl/UAnEdz}} & \hspace{0.5cm} A Social Media driven Intelligent Ad-Targeting framework using Geo-profiling. The idea is to tag all geo-location enabled tweets(available publicly) with semantic categories (say sports, politics etc.) and their sentiment (positive, neutral, negative) using text mining. To enable serving of Ads, they also need to be tagged using same categories based on their content. \\
        \href{http://github.com/debjyoti385/dartnews}{Dartnews} -\href{https://goo.gl/f8imWH}{\textit{https://goo.gl/f8imWH}}  & \hspace{0.5cm} A street news browsing application, with an interactive GIS interface. News is organized by locality and topic. The user can explore based on topics (eg crime, politics etc) and geolocation. We used Context Dependent Geoparsing, where we attempt to find out which location is relevant to the News. Across all the topics, we saw at least 87\% accuracy of topic prediction, and at least 80\% accuracy of location prediction.\\
%       \href{http://metonym.in}{Metonym (metonym.in)} & \hspace{1cm}  A synonym based vocabulary learner website built with d3.js, php and dropwizard framework at the backend. For demo use username: "demo" \& password: "12345678"  \\        
%Elliptic Curve Diffie-Hellman Key Agreement Protocol    & We here proposed an efficient Elliptic curve Diffie-Hellman two-party key agreement protocol using public key authentication based on Elliptic Curve Group. We presented an improved version of that protocol which uses Converse of Fermat’s theorem with eliminated charmichael numbers. Also efficiently generating primitive roots for D-H key.\\
%Real time discrimination of Speech and Music & Real-time speaker segmentation is required in many applications, such as speaker tracking in real- time news-video segmentation and classification, or real-time speaker adapted speech. After feature extraction, using Low Short Term Energy Ratio (LSTER) the input digital audio stream is classified into speech and music. Here we have used DSK6713 Digital Signal Processing Kit.\\

\end{tabular}

\section{Hackathons}
\begin{tabular}[t]{@{}p{\textwidth-\rcollength-190pt}p{\rcollength+180pt}}
        \textbf{[2016]} EMC2 Code & \href{https://github.com/codedellemc/mars-challenge}{Mars Challenge} Hackathon Winner (around 45 teams) \\[5pt]
        \textbf{[2015]} Goldman Sachs & Air Quality Hackathon 1st Runners-up  (around 30 teams)\\[5pt]
        \textbf{[2014]} \href{http://www.inmobi.com/}{InMobi} & Freedom Hack Worldwide Hackathon 1st Runners-up (around 160 teams)   \\[5pt]
        \textbf{[2013]} \href{http://www.yahoo.com/}{Yahoo} & Yahoo HackU 2013 Hackathon Winner (around 40 teams)   
%    \textbf{Scripts and}  & \LaTeX{}, Shell script, Javascript, \href{http://en.wikipedia.org/wiki/Pig_(programming_tool)}{Pig}, \href{http://my.vertica.com/docs/7.0.x/HTML/index.htm}{VSQL},\\
%    \textbf{Query languages} & HiveQL (Custom UDFs)  \\[5pt]
%    \textbf{Databases} & MySQL, \href{http://www.vertica.com/about/}{HP Vertica}, HBase\\[5pt]
%    \textbf{Tech tools}  & MapReduce2, \href{https://storm.apache.org/}{Apache Storm}, gdb, Vim, OhMyZsh, \\ 
%    \textbf{and Softwares} & IntellijIDEA, Maven\\[5pt]
%    \textbf{Tidbits} & CSS3, HTML5, Data Visualization (\href{http://d3js.org/}{d3.js}, \href{http://threejs.org/}{Three.js}),\\ 
%                     & \href{https://github.com/azkaban/azkaban}{Azkaban2} \& Oozie exec engine\\[5pt]
\end{tabular}

\section{Skills}
\begin{tabular}[t]{@{}p{\textwidth-\rcollength-190pt}p{\rcollength+180pt}}
    \textbf{Programming}  & Python, Java, C, C++, Javascipt, \href{http://d3js.org/}{D3.js}, \href{http://threejs.org/}{Three.js}\\[5pt]
\textbf{Databases} \textit{etc.} & MySQL, \href{http://www.vertica.com/about/}{HP Vertica}, HiveQL, VSQL, Apache Pig, HBase, MR2, Apache Storm\\[5pt]
        \textbf{Tidbits} & \LaTeX{}, Shell script, gdb, CSS3, HTML5,  Maven, Vim, Azkaban2 \textit{etc.} \\ 
\end{tabular}




\section{Achievements} 
\begin{innerlist}
\item[\textbf{[2013]}] Ranked 3rd out of 39 M.Tech students of CSE department in Indian of Institute Technology, Kanpur 
\item[\textbf{[2012]}]  Secured All India Rank 228 in GATE 2012 among 0.16 million participants and 223 in 2011 among 0.15 million participants of Computer Science \& Information Technology department. 
\item[\textbf{[2011]}]  Achieved All India Rank of 7 in Indian Space Research Organization (ISRO) recruitment exam for Junior Research Fellow among 0.2 million candidates.
%\item[\textbf{[2010]}] 2nd Rank in Manual Robo-race competition \& 2nd Rank in Robo-Olympic an innovative track based racing competition in Bits to Bytes Tech-fest, Bengal Institute of Technology. 
\end{innerlist}


%\section{Relevant Courses}
%\begin{tabular}[t]{@{}p{\textwidth-\rcollength-40pt}p{\rcollength+40pt}}
% Mathematics and algorithms for CS & Indexing and searching techniques in databases \\
%    Topics in Internet technologies &  Modern cryptology \\
%    Advanced Computer Architecture & Software Engineering \\
%    Computer networks & Digital systems design \\
%    Microprocessor and microcontroller & Data mining (Audited) \\
%    Parallel algorithms & Data structures and algorithms \\
%    Database management systems & Operating Systems \\
%\end{tabular}

%\section{Skills}
%\begin{tabular}[t]{@{}p{\textwidth-\rcollength-80pt}p{\rcollength+80pt}}
%    \textbf{Programming}  & C, C++, Java, Python\\[5pt]
%    \textbf{Scripts and}  & \LaTeX{}, Shell script, Javascript, \href{http://en.wikipedia.org/wiki/Pig_(programming_tool)}{Pig}, \href{http://my.vertica.com/docs/7.0.x/HTML/index.htm}{VSQL},\\
%    \textbf{Query languages} & HiveQL (Custom UDFs)  \\[5pt]
%    \textbf{Databases} & MySQL, \href{http://www.vertica.com/about/}{HP Vertica}, HBase\\[5pt]
%    \textbf{Tech tools}  & MapReduce2, \href{https://storm.apache.org/}{Apache Storm}, gdb, Vim, OhMyZsh, \\ 
%    \textbf{and Softwares} & IntellijIDEA, Maven\\[5pt]
%    \textbf{Tidbits} & CSS3, HTML5, Data Visualization (\href{http://d3js.org/}{d3.js}, \href{http://threejs.org/}{Three.js}),\\ 
%                     & \href{https://github.com/azkaban/azkaban}{Azkaban2} \& Oozie exec engine\\[5pt]
%\end{tabular}

%\section{Position of Responsibilities}
%\begin{innerlist}
%\item[\textbf{[2012]}] Member and organizer of SIGDATA, a Special Interests Group on DATA which meets weekly to discuss techniques related to data management and mining at IIT Kanpur.
%\item[\textbf{[2012-2013]}] Teaching assistant for Software Engineering (CS455) course at IIT Kanpur.
%\item[\textbf{[2011-2012]}]Teaching assistant for Fundamentals of Computer Science (ESC101) course at IIT Kanpur.
%\item[\textbf{[2009]}]Event coordinator and Website developer for Festronix Techfest at Institute of Engineering and Management
%\end{innerlist}
%
   
    
\begin{comment}

\section{References}
%
Available upon request

\end{comment}
\end{document}

%%%%%%%%%%%%%%%%%%%%%%%%%% End CV Document %%%%%%%%%%%%%%%%%%%%%%%%%%%%%
