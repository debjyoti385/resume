%%%%%%%%%%%%%%%%%%%%%%%%%%%%%%%%%%%%%%%%%%%%%%%%%%%%%%%%%%%%%%%%%%%%%%%%
%%%%%%%%%%%%%%%%%%%%%% Simple LaTeX CV Template %%%%%%%%%%%%%%%%%%%%%%%%
%%%%%%%%%%%%%%%%%%%%%%%%%%%%%%%%%%%%%%%%%%%%%%%%%%%%%%%%%%%%%%%%%%%%%%%%

%%%%%%%%%%%%%%%%%%%%%%%%%%%%%%%%%%%%%%%%%%%%%%%%%%%%%%%%%%%%%%%%%%%%%%%%
%% NOTE: If you find that it says                                     %%
%%                                                                    %%
%%                           1 of ??                                  %%
%%                                                                    %%
%% at the bottom of your first page, this means that the AUX file     %%
%% was not available when you ran LaTeX on this source. Simply RERUN  %%
%% LaTeX to get the ``??'' replaced with the number of the last page  %%
%% of the document. The AUX file will be generated on the first run   %%
%% of LaTeX and used on the second run to fill in all of the          %%
%% references.                                                        %%
%%%%%%%%%%%%%%%%%%%%%%%%%%%%%%%%%%%%%%%%%%%%%%%%%%%%%%%%%%%%%%%%%%%%%%%%

%%%%%%%%%%%%%%%%%%%%%%%%%%%% Document Setup %%%%%%%%%%%%%%%%%%%%%%%%%%%%

% Don't like 10pt? Try 11pt or 12pt
\documentclass[8pt]{article}

% This is a helpful package that puts math inside length specifications
\usepackage{calc}
\usepackage{array}
\usepackage{comment}
\usepackage[none]{hyphenat}
% Simpler bibsection for CV sections
% (thanks to natbib for inspiration)
\makeatletter
\newlength{\bibhang}
\setlength{\bibhang}{1em}
\newlength{\bibsep}
 {\@listi \global\bibsep\itemsep \global\advance\bibsep by\parsep}
\newenvironment{bibsection}%
        {\vspace{-\baselineskip}\begin{list}{}{%
       \setlength{\leftmargin}{\bibhang}%
       \setlength{\itemindent}{-\leftmargin}%
       \setlength{\itemsep}{\bibsep}%
       \setlength{\parsep}{\z@}%
        \setlength{\partopsep}{0pt}%
        \setlength{\topsep}{0pt}}}
        {\end{list}\vspace{-.6\baselineskip}}
\makeatother

% Layout: Puts the section titles on left side of page
\reversemarginpar

%
%         PAPER SIZE, PAGE NUMBER, AND DOCUMENT LAYOUT NOTES:
%
% The next \usepackage line changes the layout for CV style section
% headings as marginal notes. It also sets up the paper size as either
% letter or A4. By default, letter was used. If A4 paper is desired,
% comment out the letterpaper lines and uncomment the a4paper lines.
%
% As you can see, the margin widths and section title widths can be
% easily adjusted.
%
% ALSO: Notice that the includefoot option can be commented OUT in order
% to put the PAGE NUMBER *IN* the bottom margin. This will make the
% effective text area larger.
%
% IF YOU WISH TO REMOVE THE ``of LASTPAGE'' next to each page number,
% see the note about the +LP and -LP lines below. Comment out the +LP
% and uncomment the -LP.
%
% IF YOU WISH TO REMOVE PAGE NUMBERS, be sure that the includefoot line
% is uncommented and ALSO uncomment the \pagestyle{empty} a few lines
% below.
%

%% Use these lines for letter-sized paper
\usepackage[paper=letterpaper,
            %includefoot, % Uncomment to put page number above margin
            marginparwidth=.85in,     % Length of section titles
            marginparsep=.05in,       % Space between titles and text
            margin=0.55in,               % 1 inch margins
            includemp]{geometry}

%% Use these lines for A4-sized paper
%\usepackage[paper=a4paper,
%            %includefoot, % Uncomment to put page number above margin
%            marginparwidth=30.5mm,    % Length of section titles
%            marginparsep=1.5mm,       % Space between titles and text
%            margin=25mm,              % 25mm margins
%            includemp]{geometry}

%% More layout: Get rid of indenting throughout entire document
\setlength{\parindent}{0in}

%% This gives us fun enumeration environments. compactitem will be nice.
\usepackage{paralist}
\usepackage{fontspec}
\usepackage{fontawesome}


%% Reference the last page in the page number
%
% NOTE: comment the +LP line and uncomment the -LP line to have page
%       numbers without the ``of ##'' last page reference)
%
% NOTE: uncomment the \pagestyle{empty} line to get rid of all page
%       numbers (make sure includefoot is commented out above)
%
\usepackage{fancyhdr,lastpage}
\pagestyle{fancy}
%\pagestyle{empty}      % Uncomment this to get rid of page numbers
\fancyhf{}\renewcommand{\headrulewidth}{0pt}
\fancyfootoffset{\marginparsep+\marginparwidth}
\newlength{\footpageshift}
\setlength{\footpageshift}
          {0.5\textwidth+0.5\marginparsep+0.5\marginparwidth-2in}
\lfoot{\hspace{\footpageshift}%
       \parbox{4in}{\, \hfill %
                    \arabic{page} of \protect\pageref*{LastPage} % +LP
%                    \arabic{page}                               % -LP
                    \hfill \,}}

% Finally, give us PDF bookmarks
\usepackage{color,hyperref}
\definecolor{darkblue}{rgb}{0.0,0.0,0.3}
\hypersetup{colorlinks,breaklinks,
            linkcolor=darkblue,urlcolor=darkblue,
            anchorcolor=darkblue,citecolor=darkblue}

%%%%%%%%%%%%%%%%%%%%%%%% End Document Setup %%%%%%%%%%%%%%%%%%%%%%%%%%%%


%%%%%%%%%%%%%%%%%%%%%%%%%%% Helper Commands %%%%%%%%%%%%%%%%%%%%%%%%%%%%

% The title (name) with a horizontal rule under it
%
% Usage: \makeheading{name}
%
% Place at top of document. It should be the first thing.
\newcommand{\makeheading}[1]%
        {\hspace*{-\marginparsep minus \marginparwidth}%
         \begin{minipage}[t]{\textwidth+\marginparwidth+\marginparsep}%
                {\large \bfseries #1}\\[-0.15\baselineskip]%
                 \rule{\columnwidth}{1pt}%
         \end{minipage}}

% The section headings
%
% Usage: \section{section name}
%
% Follow this section IMMEDIATELY with the first line of the section
% text. Do not put whitespace in between. That is, do this:
%
%       \section{My Information}
%       Here is my information.
%
% and NOT this:
%
%       \section{My Information}
%
%       Here is my information.
%
% Otherwise the top of the section header will not line up with the top
% of the section. Of course, using a single comment character (%) on
% empty lines allows for the function of the first example with the
% readability of the second example.
\renewcommand{\section}[2]%
        {\pagebreak[2]\vspace{0.7\baselineskip}%
         \phantomsection\addcontentsline{toc}{section}{#1}%
         \hspace{0in}%
         \marginpar{
         \raggedright \scshape #1}#2}

% An itemize-style list with lots of space between items
\newenvironment{outerlist}[1][\enskip\textbullet]%
        {\begin{itemize}[#1]}{\end{itemize}%
         \vspace{-.6\baselineskip}}

% An environment IDENTICAL to outerlist that has better pre-list spacing
% when used as the first thing in a \section
\newenvironment{lonelist}[1][\enskip\textbullet]%
        {\vspace{-\baselineskip}\begin{list}{#1}{%
        \setlength{\partopsep}{0pt}%
        \setlength{\topsep}{0pt}}}
        {\end{list}\vspace{-.6\baselineskip}}

% An itemize-style list with little space between items
\newenvironment{innerlist}[1][\enskip\textbullet]%
        {\begin{compactitem}[#1]}{\end{compactitem}}

% An environment IDENTICAL to innerlist that has better pre-list spacing
% when used as the first thing in a \section
\newenvironment{loneinnerlist}[1][\enskip\textbullet]%
        {\vspace{-\baselineskip}\begin{compactitem}[#1]}
        {\end{compactitem}\vspace{-.6\baselineskip}}

% To add some paragraph space between lines.
% This also tells LaTeX to preferably break a page on one of these gaps
% if there is a needed pagebreak nearby.
\newcommand{\blankline}{\quad\pagebreak[2]}

% Uses hyperref to link DOI
\newcommand\doilink[1]{\href{http://dx.doi.org/#1}{#1}}
\newcommand\doi[1]{doi:\doilink{#1}}


%%%%%%%%%%%%%%%%%%%%%%%% End Helper Commands %%%%%%%%%%%%%%%%%%%%%%%%%%%

%%%%%%%%%%%%%%%%%%%%%%%%% Begin CV Document %%%%%%%%%%%%%%%%%%%%%%%%%%%%

\begin{document}
\makeheading{\Large{Debjyoti Paul} {\color{blue}\small\hspace{13.3cm} July, 2022}}
% \vspace{-1cm}
%
% \vspace{.3cm}

\section{Contact}
%
% NOTE: Mind where the & separators and \\ breaks are in the following
%       table.
%
% ALSO: \rcollength is the width of the right column of the table
%       (adjust it to your liking; default is 1.85in).
%
\newlength{\rcollength}\setlength{\rcollength}{3in}%
%
% \begin{tabular}[t]{@{}p{\textwidth-\rcollength }p{\rcollength}}
% 130 S 800 E APT 101  & \faHome ~ \href{http://www.cs.utah.edu/~deb}{\url{www.cs.utah.edu/\~deb}}  \\
%         Salt Lake City   & \faEnvelope ~ \href{mailto:deb@cs.utah.edu}{\url{deb@cs.utah.edu}}\\
%         ZIP: 84102  &  \faLinkedin ~ \href{https://www.linkedin.com/in/debjyotipaul385/}{\url{/in/debjyotipaul385/}} \\
%         University of Utah & \faGithub ~ \href{https://github.com/debjyoti385}{\url{www.github.com/debjyoti385}} \\
%         USA     &  \faPhone ~ \texttt{+1 385 E1E-7219} \\
% \end{tabular}
\begin{tabular}[t]{@{}p{\textwidth-\rcollength }p{\rcollength}}
\faEnvelope ~ \href{mailto:dr.debjyotipaul@gmail.com}{\url{dr.debjyotipaul@gmail.com}}  & \faHome ~ \href{http://www.debjyotipaul.in}{\url{www.debjyotipaul.in}}  \\
        % Salt Lake City   & \faEnvelope ~ \href{mailto:deb@cs.utah.edu}{\url{deb@cs.utah.edu}}\\
        \faLinkedin ~ \href{https://www.linkedin.com/in/debjyotipaul385/}{\url{/in/debjyotipaul385/}} & \faGithub ~ \href{https://github.com/debjyoti385}{\url{www.github.com/debjyoti385}}
        % ZIP: 84102  &  \faLinkedin ~ \href{https://www.linkedin.com/in/debjyotipaul385/}{\url{/in/debjyotipaul385/}} \\
        % University of Utah & \faGithub ~ \href{https://github.com/debjyoti385}{\url{www.github.com/debjyoti385}} \\
        % USA     &  \faPhone ~ \texttt{+1 385 E1E-7219} \\
\end{tabular}


\section{ Research Interests}
Automatic Speech Recognition System, Natural Language Processing and Sentiment Analysis, Graph \\Representation Learning, Spatio-Temporal Public Health Analysis.
%
% Spatio-Temporal Data Analysis, Social Media Analysis, Healthcare Analytics, Machine Learning, \\Representation Learning, Data Visualization
%


\section{Work Experience}
\textbf{[March 2020 - \textit{Present}] Meta, Inc.}, Menlo Park, United States
\vspace{-2mm}
\begin{outerlist}
\item[] [Feb 2022 ~-~\textit{Present}\hspace{2mm}] \textbf{Senior Research Scientist} \vspace{-2mm}
\item[] [Mar 2020 - Feb 2022]  \textbf{Research Scientist} \vspace{-2mm}
\item[] Scientist in AI Speech team responsible for creating, improving, and deploying state-of-the-art Automated Speech Recognition, Inverse Text Normalization and Punctuation-Capitalization models.
\end{outerlist}
\vspace{1mm}

\textbf{[Summer 2019] Alibaba Group., AI Research Intern}, Sunnyvale, United States
\vspace{-2mm}
\begin{outerlist}
\item[] Workload-aware database tuning with Machine Learning.
\end{outerlist}
\vspace{1mm}

\textbf{[Summer 2018] Facebook, Inc., Research Scientist Intern}, Seattle, United States
\vspace{-2mm}
\begin{outerlist}
\item[] Improving metrics for Facebook's Search Ranking via rated label inferences with Machine Learning.
\end{outerlist}
\vspace{1mm}

\textbf{[Summer 2017] Amazon AI Research, Research Intern}, New York, United States
\vspace{-2mm}
\begin{outerlist}
\item[] Hyperparameter optimization techniques on {\sc MxNet}.
\end{outerlist}
\vspace{1mm}

\textbf{[2013-2015] \href{http://www.flipkart.com/}{Flipkart}, Data Engineer}, Bangalore, India
\vspace{-2mm}
\begin{outerlist}
\item[] Served as data engineer at data platform team \href{http://www.flipkart.com}{\emph{flipkart.com}}, India's biggest e-commerce company.\\ [-6mm]
\item[] Facilitated scalable environment for Big data analytics with processing pipelines for batch and stream.
\end{outerlist}

\section{ Education}
%
\textbf{[2015-2020] Ph.D.,} \href{http://www.cs.utah.edu/}{\textbf{University of Utah, School of Computing}},
Salt Lake City, Utah, USA\\
\vspace{-2mm}
\hspace{0.5cm}
\begin{tabular}[t]{@{}p{24pt}p{\rcollength+200pt}}
        Major    &: \href{http://cs.utah.edu/}{{Computer Science \& Engineering}} \\
        Advisor &: \href{http://www.cs.utah.edu/~lifeifei/}{\em Prof. Feifei Li} \\
        Thesis   &: \emph{Data-Driven Spatio-Temporal Analysis for Multimode Data} \\
        GPA      &: {\em 3.96/4.0} \\[5pt]
\end{tabular}
\vspace{0.1cm}

\textbf{[2011-2013] Masters of Technology,} \href{http://www.iitk.ac.in/}{\textbf{Indian Institute of Technology Kanpur}},
Kanpur, India\\
\vspace{-2mm}
\hspace{0.5cm}
\begin{tabular}[t]{@{}p{24pt}p{\rcollength+200pt}}
        Major    &: \href{http://cse.iitk.ac.in/}{\em Computer Science \& Engineering} \\
        Advisor  &: \href{}{\em Late Professor  Sanjeev K Aggarwal}\\
        Thesis   &: \emph{Multi-constraint Job scheduling in Grid Computing} \\
        GPA      &: {\em 8.67/10.0 (Rank: 3) } \\[5pt]
\end{tabular}
\vspace{0.1cm}

\textbf{[2007-2011] Bachelors of Technology}, \href{http://www.wbut.ac.in/}{\textbf{West Bengal University of Technology}},  Kolkata, India
\vspace{-2mm}
\hspace{0.6cm}
\begin{tabular}[t]{@{}p{24pt}p{\rcollength+200pt}}
        Major   &: \href{http://cse.iitk.ac.in/}{\em Computer Science \& Engineering} \\
        College &: \href{http://www.iemcal.com/}{\emph{Institute of Enginering and Management}}\\
        CGPA    &: {\em 8.93/10 (Rank: $< 10$)} \\[5pt]
\end{tabular}
\vspace{0.1cm}



\section{Theses}
\textbf{[2020]  Ph.D. Thesis, Advisor: \href{http://www.cs.utah.edu/~lifeifei}{Dr. Feifei Li}}\\
{\em \textbf{Data-Driven Spatio-Temporal Analysis for Multimode Data.}}\\
The objective of this thesis is to build a framework that enables a
data-driven approach for spatio-temporal analysis for multimode data with AI models and facilitate workload-aware support for effective large-scale processing.\vspace{0.2cm}\\
\textbf{[2013]  M.Tech Thesis, Advisor: \href{http://www.cse.iitk.ac.in/users/ska}{Late Dr. Sanjeev Kumar Aggarwal}}\\
{\em \textbf{Multi-constraint Job scheduling in Grid Computing}}\\
% \vspace{0.05cm}\\
The objective is to efficiently schedule Jobs on Grid Computing to achieve maximum utilization of resources with energy efficient approach. Modeling real world computing \& storage grid on Multi-objective Evolutionary Algorithm satisfying hard constraints on jobs constraints, resources constraints, and soft constraints on cost, and energy consumption.
%on NSGA-II. Local optimization using Pareto optimal front, global optimization by applying mutation on population on search space. Also introduced job grouping technique with constraints to accumulate fine grained jobs which keeps processing time as low as possible.
% \vspace{-2mm}
% \begin{outerlist}
% % \item[] \textbf{Multi-constraint Job scheduling in Grid Computing} \\[-1.7em]
% \item[] The objective is to efficiently schedule Jobs on Grid to achieve maximum utilization of resources with energy efficient approach. Modeling real world computing \& storage grid on Multi-objective Evolutionary Algorithm satisfying hard constraints on jobs constraints, resources constraints, and soft constraints on cost, and energy consumption on NSGA-II. Local optimization using Pareto optimal front, global optimization by applying mutation on population on search space. Also introduced job grouping technique with constraints to accumulate fine grained jobs which keeps processing time as low as possible.
% \end{outerlist}


%\textbf{[2010] Jadavpur University,} \emph{System Administration}, Trainee,\\
%    Kolkata, India
%    \begin{innerlist}
%    \item Practical experience with servers and networking devices and on various aspect of system administration
%    \end{innerlist}

\section{Publications}
\\
\begin{bibsection}
\item[+] Improving Data Driven Inverse Text Normalization using Data Augmentation, Laxmi Pandey, \textbf{Debjyoti Paul}, Pooja Chitkara, Yutong Pang (In Review).
\item[+] Improving Data Driven Inverse Text Normalization using Data Augmentation and Machine Translation,  Debjyoti Paul, Yutong Pang, Sray Chen, Xuedong Zhang, \emph{Interspeech 2022 Show and Tell}, Incheon, Korea.
\item[+] Database Workload Characterization with Query Plan Encoders, \textbf{Debjyoti Paul}, Jie Cao, Feifei Li, Vivek Srikumar, To appear in {\em 48th International Conference on Very Large Data Bases (VLDB 2022)}.
\item[+] Semantic Embedding for Regions of Interest, \textbf{Debjyoti Paul}, Feifei Li, Jeff M. Phillips, \emph{The International Journal on Very Large Data Bases (VLDBJ 2021).} \textit{DOI:} \href{https://doi.org/10.1007/s00778-020-00647-0}{10.1007/s00778-020-00647-0}.
\item[+] Bursty Event Detection Throughout Histories, \textbf{Debjyoti Paul}, Yanqing Peng, Feifei Li, \emph{35th IEEE International Conference on Data Engineering (ICDE 2019)}, Macau, China, 2019. \textit{DOI:} \href{https://dx.doi.org/10.1109/ICDE.2019.00124}{10.1109/ICDE.2019.00124}
\item[+] Compass: Spatio Temporal Sentiment Analysis of US Election, \textbf{Debjyoti Paul}, Feifei Li, Murali Krishna Teja, Xin Yu, Richie Frost,  \textit{What twitter says!}, \emph{23rd SIGKDD Conference on Knowledge Discovery and Data Mining (SIGKDD)}, Aug 13-17, 2017, Halifax, Canada. \textit{DOI:} \href{https://dx.doi.org/10.1145/3097983.3098053}{10.1145/3097983.3098053}.
\item[+] \href{https://www.cs.utah.edu/~deb/aipro/}{AI Pro}: Data Processing Framework for AI Models, \textbf{Debjyoti Paul}, Richie Frost, Feifei Li, \emph{35th IEEE International Conference on Data Engineering (ICDE)}, Macau, China, 2019.
\item[+] Geotagged US Tweets as Predictors of County-Level Health Outcomes, 2015–2016, Quynh C. Nguyen, Matt McCullough, Hsien-wen Meng, \textbf{Debjyoti Paul}, Dapeng Li,  \emph{American Journal of Public Health (AJPH), September, 2017}, \textit{DOI:} \href{https://dx.doi.org/10.2105/AJPH.2017.303993}{10.2105/AJPH.2017.303993}.
\item[+] Social media indicators of the food environment and state health outcomes, Quynh C. Nguyen, Hsien-wen Meng, Dapeng. Li, Matt McCullough, \textbf{Debjyoti Paul}, Kanokvimankul. P, Nguyen. T, Li. Feifei, \emph{ American Public Health Association (APHA), 148, 120-128.}, 2017, \textit{DOI:} \href{https://dx.doi.org/10.1016/j.puhe.2017.03.013}{10.1016/j.puhe.2017.03.013}.
\item[+] Multi-objective Evolution based Dynamic Job Scheduler in Grid, \textbf{Debjyoti Paul}, Sanjeev K. Aggarwal,  \emph{The 8th International Conference on Complex, Intelligent, and Software Intensive Systems (CISIS)}, IEEE, July 2nd - 4th, 2014, Birmingham, UK. \textit{DOI:} \href{https://dx.doi.org/10.1109/CISIS.2014.50}{10.1109/CISIS.2014.50}.
\item[+] RCached-tree: An Index Structure for Efficiently Answering Popular Queries, Manash Pal, Arnab Bhattacharya, \textbf{Debjyoti Paul}, \emph{ACM International Conference on Information and Knowledge Management (CIKM 2013)}, Oct. 27 - Nov. 1, 2013, San Francisco, CA, USA. \textit{DOI:}  \href{https://dx.doi.org/10.1145/2505515.2507817}{10.1145/2505515.2507817}.
\item[+] Lightweight Security Enhancement Protocol for Radio Frequency Identification (RFID),  \textbf{Debjyoti Paul}, Sumana Basu, Punit Beriwal, \emph{IEEE SPSITM International Conference 2011}, Kolkata, India.
\end{bibsection}

%\section{Summer Schools}\\
%\textbf{[2012] Microsoft Research}, \emph{Summer School on Distributed Computing}\\
%Bangalore, India
%        \begin{innerlist}
%        \item Familiarization with current research work on distributed computing\\
%            \end{innerlist}

%\section{Scholarship}
%\begin{innerlist}
%\item GBP 1,700 from IIT Kanpur for Paper presentation in CISIS 2014, Birmingham, UK
%\item 96,000 INR per year from Ministry of Human Resource Development India during M.Tech.
%\item 2 years waiver of tuition fees at school for academic excellence in school.
%\end{innerlist}


%\section{Certifications}
%\begin{innerlist}
%\item[\textbf{[2011]}] Linux system administration from School of Mobile Computing and Communication Jadavpur University.
%\item[\textbf{[2010]}] Microsoft certification on Database Fundamentals.
%\end{innerlist}

\vspace{0.2cm}
\section{Position of Responsibilities}\\
\begin{innerlist}
\item[\textbf{[2021]}] Program Committee Member for 47th International Conference on Very Large Data Bases (VLDB) 2021, Industrial Track.
\item[\textbf{[2021]}] Program Committee Member for 37th IEEE International Conference on Data Engineering (ICDE) 2021, Industrial Track.
\item[\textbf{[2018-2021]}] Reviewer: IEEE TKDE 2018, IEEE TKDE 2019, IEEE ICDE 2021, PVLDB 2021.
\item[\textbf{[2016-2019]}] External Reviewer: EDBT 2016, PVLDB 2019.
\item[\textbf{[2015-2018]}] Teaching Mentor: Natural Language Processing (CS6340), Data Mining (CS 6140) at the University of Utah. USA.
\item[\textbf{[2012-2013]}] Member and organizer of SIGDATA, a Special Interests Group on data management and mining at the IIT Kanpur, India.
\item[\textbf{[2011-2013]}] Teaching assistant: Software Engineering (CS455), Fundamentals of Computer Science (ESC101) at IIT Kanpur.
\end{innerlist}

% \vspace{2mm}
% \section{Projects \href{https://www.cs.utah.edu/~deb/\#projects}{\scriptsize (For~ more...  {\scriptsize \faHome/\#projects)}}}
% \begin{tabular}[t]{@{}>{\raggedright\arraybackslash}p{\textwidth-\rcollength-150pt}p{\rcollength+150pt}}
%         \href{http://estorm.org}{Compass } ~~~~~~\href{http://estorm.org}{\textit{estorm.org}} & \hspace{0.5cm} Compass is a framework for large scale Spatio-temporal Sentiment Analysis on social data. Example projects supported by Compass: Popularity of Political Parties for the Presidential Election 2016, Social Media indicators and social health outcomes. Read publications for details.\\[0.1cm]
%         Event Aggregator & \hspace{0.5cm}  This project finds events from news articles, categorize them and gathers all articles talking about same event to a set. Each set of articles is referred as Event Entity. Information extraction process is applied to extract more information related to it. The code can process large scale data. The code will be made public later. \\[0.1cm]
%         \href{https://github.com/debjyoti385/QuestionAnswerNLP}{QuestionAnswering} \href{https://goo.gl/QGfW2y}{\textit{goo.gl/QGfW2y}} & \hspace{0.5cm} A Natural Language Processing project focussed on closed domain Question Answering System. The project has Question Classifier, Sentence Similarity, Answer formulation and Coref resolution modules. The system has \emph{Recall} of  63\% and \emph{F-score} of 43\%.   \\[0.1cm]
%         \href{http://github.com/debjyoti385/intelliad}{Intelliad} \href{https://goo.gl/UAnEdz}{\textit{goo.gl/UAnEdz}} & \hspace{0.5cm} A Social Media driven Intelligent Ad-Targeting framework using Geo-profiling. The idea is to tag all geo-location enabled tweets(available publicly) with semantic categories (say sports, politics etc.) and their sentiment (positive, neutral, negative) using text mining. To enable serving of Ads, they also need to be tagged using same categories based on their content. \\[0.2cm]
%         \href{http://debjyoti385.github.io/AirQuality/}{AirQuality} -\href{https://goo.gl/F8twhn}{\textit{https://goo.gl/F8twhn}}  & \hspace{0.5cm}  Goldman Sachs Hackathon - This analyze the air quality of Utah state from real world dataset. Data warehousing and cubing to enable interactive visualization framework, which helped in determining pattern of Pollution and its reason. \\
% \end{tabular}
% \begin{tabular}[t]{@{}>{\raggedright\arraybackslash}p{\textwidth-\rcollength-150pt}p{\rcollength+150pt}}
%         \href{https://goo.gl/wMOl4C}{Dartnews} \href{https://goo.gl/wMOl4C}{\textit{goo.gl/wMOl4C}}  & \hspace{0.5cm} A street news browsing application, with an interactive GIS interface. News is organized by locality and topic. The user can explore based on topics (eg crime, politics etc) and geolocation. We used Context Dependent Geoparsing, where we attempt to find out which location is relevant to the News. Across all the topics, we saw at least 87\% accuracy of topic prediction, and at least 80\% accuracy of location prediction.\\[0.1cm]
%         \href{https://goo.gl/ovcWVT}{QuakeAnalysis} \href{https://goo.gl/ovcWVT}{\textit{goo.gl/ovcWVT}} & \hspace{0.5cm}  This project is based on the seismic activity across world which widely varies in characteristic and patterns. We have found some distinguish patterns among the seismic activities and present them in an insightful manner. Check \href{https://goo.gl/ovcWVT}{demo} and wiki of \href{https://bitbucket.org/debjyotipaul385/quakeanalysis/wiki/Exploration}{Exploration} \& \href{https://bitbucket.org/debjyotipaul385/quakeanalysis/wiki/Analysis}{Analysis}\\
%         \href{https://github.com/musicatlas/}{MusicAtlas}
%         % \href{https://goo.gl/UH54UP}{\textit{goo.gl/UH54UP}}
%         & \hspace{0.5cm}  This website is designed for music lovers to learn about music based on countries. This is a unique tool to explore and analyze the trend of music based on time frame, genre and artists. Almost all data from 19th century to till date. \\[0.1cm]
% %        RCached Tree ~~~~~ \href{https://doi.org/10.1145/2505515.2507817}{\textit{doi.org-rcached-tree}} & \hspace{0.5cm}  The objective is to speed up performance of point, range and kNN search queries for popular queries in databases. We came up with a variant of the R-Tree indexing structure with some features like caching at each node for popular queries.\\[0.1cm]

% %       \href{http://metonym.in}{Metonym (metonym.in)} & \hspace{1cm}  A synonym based vocabulary learner website built with d3.js, php and dropwizard framework at the backend. For demo use username: "demo" \& password: "12345678"  \\
% % Elliptic Curve Diffie-Hellman Key Agreement Protocol    & We here proposed an efficient Elliptic curve Diffie-Hellman two-party key agreement protocol using public key authentication based on Elliptic Curve Group. We presented an improved version of that protocol which uses Converse of Fermat’s theorem with eliminated charmichael numbers. Also efficiently generating primitive roots for D-H key.\\
% Real time discrimination of Speech and Music & Real-time speaker segmentation is required in many applications, such as speaker tracking in real- time news-video segmentation and classification, or real-time speaker adapted speech. After feature extraction, using Low Short Term Energy Ratio (LSTER) the input digital audio stream is classified into speech and music. Here we have used DSK6713 Digital Signal Processing Kit.\\
% \end{tabular}
% \textit{Find more projects at \href{https://www.cs.utah.edu/~deb/\#projects}{https://www.cs.utah.edu/\textasciitilde deb/\#project}}


%\section{Skills}
%\begin{tabular}[t]{@{}p{\textwidth-\rcollength-190pt}p{\rcollength+210pt}}
%    \textbf{Programming}  & Python, Java, C, C++, Javascipt, \href{http://d3js.org/}{D3.js}, \href{http://threejs.org/}{Three.js}\\[5pt]
%\textbf{Databases} \textit{etc.} & MySQL, \href{http://www.vertica.com/about/}{HP Vertica}, HiveQL, VSQL, Apache Pig, HBase, MR2, Apache Storm\\[5pt]
%        \textbf{Tidbits} & \LaTeX{}, Shell script, gdb, CSS3, HTML5,  Maven, Vim, Azkaban2 \textit{etc.} \\
%\end{tabular}



% \section{Achievements}
% \begin{innerlist}
% %\item[\textbf{[2013]}] Ranked 3rd out of 39 M.Tech students of CSE department in Indian of Institute Technology, Kanpur
% \item[\textbf{[2015-2020]}] \textbf{Ph.D. Fellowship} at University of Utah.
% \item[\textbf{[2012]}] \textbf{All India Rank 7} in Indian Space Research Organization (ISRO) Junior Research Fellow Exam.
% \item[\textbf{[2011]}]  Secured \textbf{All India Rank of 7} in Indian Space Research Organization (ISRO)'s Junior Research Fellow exam out of 200,000+ candidates.
% \item[\textbf{[2012]}]  Secured All India Rank 228 out of 160,000+ participants and rank 223 out of 150,000+ participants in in Graduate Aptitude Test for Engineering (GATE) in 2012 and 2011 respectively in Computer Science.
% % \item[\textbf{[2011, 2012]}] Graduate Aptitude Test for Engineering (GATE), Computer Science: \textbf{99.96\%tile and 99.98\%tile.}
% % \item[\textbf{[2011]}]  Achieved All India Rank of 7 in Indian Space Research Organization (ISRO) recruitment exam for Junior Research Fellow among 0.2 million candidates.
% %\item[\textbf{[2010]}] 2nd Rank in Manual Robo-race competition \& 2nd Rank in Robo-Olympic an innovative track based racing competition in Bits to Bytes Tech-fest, Bengal Institute of Technology.
% \end{innerlist}
%


%\section{Relevant Courses}
%\begin{tabular}[t]{@{}p{\textwidth-\rcollength-40pt}p{\rcollength+40pt}}
% Mathematics and algorithms for CS & Indexing and searching techniques in databases \\
%    Topics in Internet technologies &  Modern cryptology \\
%    Advanced Computer Architecture & Software Engineering \\
%    Computer networks & Digital systems design \\
%    Microprocessor and microcontroller & Data mining (Audited) \\
%    Parallel algorithms & Data structures and algorithms \\
%    Database management systems & Operating Systems \\
%\end{tabular}

%\section{Skills}
%\begin{tabular}[t]{@{}p{\textwidth-\rcollength-80pt}p{\rcollength+80pt}}
%    \textbf{Programming}  & C, C++, Java, Python\\[5pt]
%    \textbf{Scripts and}  & \LaTeX{}, Shell script, Javascript, \href{http://en.wikipedia.org/wiki/Pig_(programming_tool)}{Pig}, \href{http://my.vertica.com/docs/7.0.x/HTML/index.htm}{VSQL},\\
%    \textbf{Query languages} & HiveQL (Custom UDFs)  \\[5pt]
%    \textbf{Databases} & MySQL, \href{http://www.vertica.com/about/}{HP Vertica}, HBase\\[5pt]
%    \textbf{Tech tools}  & MapReduce2, \href{https://storm.apache.org/}{Apache Storm}, gdb, Vim, OhMyZsh, \\
%    \textbf{and Softwares} & IntellijIDEA, Maven\\[5pt]
%    \textbf{Tidbits} & CSS3, HTML5, Data Visualization (\href{http://d3js.org/}{d3.js}, \href{http://threejs.org/}{Three.js}),\\
%                     & \href{https://github.com/azkaban/azkaban}{Azkaban2} \& Oozie exec engine\\[5pt]
%\end{tabular}

%

%\section{References}
%%
%\textit{For skillsets please visit homepage \href{https://www.cs.utah.edu/~deb/\#skills}{https://www.cs.utah.edu/\textasciitilde deb/\#skills}.\\
%%All references are provided upon requests.
%}
%


\vspace{0.2cm}
\section{Hackathon Trophies}
\begin{tabular}[t]{@{}p{\textwidth-\rcollength-120pt}p{\rcollength+190pt}}
        \textbf{[2017]} HackTheU & HackTheU MLH Hackathon Winner  (around 40 teams) \\[5pt]
        \textbf{[2016]} EMC2 Code & \href{https://github.com/codedellemc/mars-challenge}{Mars Challenge} Hackathon Winner (around 45 teams) \\[5pt]
        \textbf{[2015]} Goldman Sachs & Air Quality Hackathon 1st Runners-up  (around 30 teams)\\[5pt]
        \textbf{[2014]} \href{http://www.inmobi.com/}{InMobi} & Freedom Hack Worldwide Hackathon 1st Runners-up (around 160 teams)   \\[5pt]
        \textbf{[2013]} \href{http://www.yahoo.com/}{Yahoo} & Yahoo HackU 2013 Hackathon Winner (around 40 teams)
%    \textbf{Scripts and}  & \LaTeX{}, Shell script, Javascript, \href{http://en.wikipedia.org/wiki/Pig_(programming_tool)}{Pig}, \href{http://my.vertica.com/docs/7.0.x/HTML/index.htm}{VSQL},\\
%    \textbf{Query languages} & HiveQL (Custom UDFs)  \\[5pt]
%    \textbf{Databases} & MySQL, \href{http://www.vertica.com/about/}{HP Vertica}, HBase\\[5pt]
%    \textbf{Tech tools}  & MapReduce2, \href{https://storm.apache.org/}{Apache Storm}, gdb, Vim, OhMyZsh, \\
%    \textbf{and Softwares} & IntellijIDEA, Maven\\[5pt]
%    \textbf{Tidbits} & CSS3, HTML5, Data Visualization (\href{http://d3js.org/}{d3.js}, \href{http://threejs.org/}{Three.js}),\\
%                     & \href{https://github.com/azkaban/azkaban}{Azkaban2} \& Oozie exec engine\\[5pt]
\end{tabular}

\section{Achievements}
\begin{innerlist}
%\item[\textbf{[2013]}] Ranked 3rd out of 39 M.Tech students of CSE department in Indian of Institute Technology, Kanpur
% \item[\textbf{[2015-2020]}] \textbf{Ph.D. Fellowship} at University of Utah.
\item[\textbf{[2012]}]  Secured \textbf{All India Rank 228} out of 160,000+ participants in Graduate Aptitude Test for Engineering (GATE) 2012, Computer Science. Secured \textbf{All India Rank 223} out of 150,000+ participants in Graduate Aptitude Test for Engineering (GATE) 2011, Computer Science.
\item[\textbf{[2011]}] \textbf{All India Rank 7} in Indian Space Research Organization (ISRO) Junior Research Fellow Exam.
% \item[\textbf{[2011]}]  Secured \textbf{All India Rank of 7} in Indian Space Research Organization (ISRO)'s Junior Research Fellow exam out of 200,000+ candidates.
% \item[\textbf{[2011]}]  Secured \textbf{All India Rank 223} out of 150,000+ participants in Graduate Aptitude Test for Engineering (GATE) 2011, Computer Science.
% \item[\textbf{[2011, 2012]}] Graduate Aptitude Test for Engineering (GATE), Computer Science: \textbf{99.96\%tile and 99.98\%tile.}
% \item[\textbf{[2011]}]  Achieved All India Rank of 7 in Indian Space Research Organization (ISRO) recruitment exam for Junior Research Fellow among 0.2 million candidates.
%\item[\textbf{[2010]}] 2nd Rank in Manual Robo-race competition \& 2nd Rank in Robo-Olympic an innovative track based racing competition in Bits to Bytes Tech-fest, Bengal Institute of Technology.
\end{innerlist}


\end{document}

%%%%%%%%%%%%%%%%%%%%%%%%%% End CV Document %%%%%%%%%%%%%%%%%%%%%%%%%%%%%
